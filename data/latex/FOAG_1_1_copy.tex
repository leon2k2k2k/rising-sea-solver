% This document was generated by underleaf.ai
\documentclass[12pt]{article}
\usepackage{amsmath, amssymb, amsthm}
\usepackage{graphicx}
\usepackage{xcolor}
\usepackage{geometry}
\usepackage{fancyhdr}
\geometry{margin=1in}
\pagestyle{fancy}
\fancyhf{}
\fancyhead[C]{\thepage}

\begin{document}

Universal property are bijections. In short, there is a unique bijection between $\mathcal{P}_{1}$ and $\mathcal{P}_{2}$ preserving the “product structure” (the maps to $\mathcal{M}$ and $\mathcal{N}$). This gives us the right to name any such product $\mathcal{M} \times \mathcal{N}$, since any two such products are uniquely identified.

This definition has the advantage that it works in many circumstances, and once we define categories, we will soon see that the above argument applies verbatim in any category to show that products, if they exist, are unique up to unique isomorphism. Even if you haven’t seen the definition of category before, you can verify that this agrees with your notion of product in some category that you have seen before (such as the category of vector spaces, or the category of manifolds).

This is handy even in cases that you understand. For example, one way of defining the product of two manifolds $\mathcal{M}$ and $\mathcal{N}$ is to cut them both up into charts, then take products of charts, then glue them together. But if I cut up the manifolds in one way, and you cut them up in another, how do we know our resulting manifolds are the “same”? We could wave our hands, or make an annoying argument about refining covers, but instead, we should just show that they are “categorical products” and hence canonically the “same” (i.e., isomorphic). We will formalize this argument in §1.2.

Another set of notions we will abstract are categories that “behave like modules”. We will want to define kernels and cokernels for new notions, and we should make sure that these notions behave the way we expect them to. This leads us to the definition of \textit{abelian categories}, first defined by Grothendieck in his Tôhoku paper \textit{[x11]}.

In this chapter, we will give an informal introduction to these and related notions, in the hope of giving just enough familiarity to comfortably use them in practice.

\section*{1.1 Categories and functors}

\textit{The introduction of the digit $0$ or the group concept was general nonsense too, and mathematics was more or less stagnating for thousands of years because nobody was around to take such childish steps…}

\begin{flushright}
— \textit{A. Grothendieck}, \textit{[x3, p. 4–5]}
\end{flushright}

\textit{Before functoriality, people lived in caves.}

\begin{flushright}
— \textit{B. Conrad}
\end{flushright}

We begin with an informal definition of categories and functors.

\subsection*{1.1.1. Categories}

A category consists of a collection of objects, and for each pair of objects, a set of morphisms (or arrows) between them. (For experts: technically, this is the definition of a \textit{locally small category}. In the correct definition, the morphisms need only form a class, not necessarily a set, but see Caution 0.3.1.) Morphisms are often informally called maps. The collection of objects of a category $\mathcal{C}$ is often denoted $\operatorname{obj}(\mathcal{C})$, but we will usually denote the collection also by $\mathcal{C}$. If $A,B \in \mathcal{C}$, then the set of morphisms from $A$ to $B$ is denoted $\operatorname{Mor}(A,B)$. A morphism is often written $\mathsf{f} \colon A \to B$, and $A$ is said to be the source of $\mathsf{f}$, and $B$ the target of $\mathsf{f}$. (Of course, $\operatorname{Mor}(A,B)$ is taken to be disjoint from $\operatorname{Mor}(A',B')$ unless $A = A'$ and $B = B'$.)

Morphisms compose as expected: there is a composition
\[
\operatorname{Mor}(B, C) \times \operatorname{Mor}(A, B) \to \operatorname{Mor}(A, C),
\]
and if $f \in \operatorname{Mor}(A, B)$ and $g \in \operatorname{Mor}(B, C)$, then their composition is denoted $g \circ f$. Composition is associative: if $f \in \operatorname{Mor}(A, B)$, $g \in \operatorname{Mor}(B, C)$, and $h \in \operatorname{Mor}(C, D)$, then
\[
h \circ (g \circ f) = (h \circ g) \circ f.
\]
For each object $A \in \mathcal{C}$, there is always an identity morphism $\operatorname{id}_A \colon A \to A$, such that when you (left- or right-)compose a morphism with the identity, you get the same morphism. More precisely, for any morphisms $f \colon A \to B$ and $g \colon B \to C$, $\operatorname{id}_B \circ f = f$ and $g \circ \operatorname{id}_B = g$. (If you wish, you may check that “identity morphisms are unique”: there is only one morphism deserving the name $\operatorname{id}_A$.) This ends the definition of a category.

We have a notion of isomorphism between two objects of a category (a morphism $f \colon A \to B$ such that there exists some — necessarily unique — morphism $g \colon B \to A$, where $f \circ g$ and $g \circ f$ are the identity on $B$ and $A$ respectively).

\subsubsection*{1.1.2. Example}
The prototypical example to keep in mind is the category of sets, denoted $\mathbf{Sets}$. The objects are sets, and the morphisms are maps of sets. (Because Russell’s paradox shows that there is no set of all sets, we did not say earlier that there is a set of all objects. But as stated in §0.3, we are deliberately omitting all set-theoretic issues.)

\subsubsection*{1.1.3. Example}
Another good example is the category $\operatorname{Vec}_k$ of vector spaces over a given field $k$. The objects are $k$-vector spaces, and the morphisms are linear transformations. (What are the isomorphisms?)

\subsubsection*{1.1.A. Unimportant Exercise}
A category in which each morphism is an isomorphism is called a groupoid. (This notion is not important in what we will discuss. The point of this exercise is to give you some practice with categories, by relating them to an object you know well.)

\begin{enumerate}
\item[(a)] A perverse definition of a group is: a groupoid with one object. Make sense of this. (Similarly, in case you care: a perverse definition of a monoid is: a category with one object.)
\item[(b)] Describe a groupoid that is not a group.
\end{enumerate}

\subsubsection*{1.1.B. Exercise}
If $A$ is an object in a category $\mathcal{C}$, show that the invertible elements of $\operatorname{Mor}(A, A)$ form a group (called the automorphism group of $A$, denoted $\operatorname{Aut}(A)$). What are the automorphism groups of the objects in Examples 1.1.2 and 1.1.3? Show that two isomorphic objects have isomorphic automorphism groups. (For readers with a topological background: if $X$ is a topological space, then the fundamental groupoid is the category where the objects are points of $X$, and the morphisms $x \to y$ are paths from $x$ to $y$, up to homotopy. Then the automorphism group of $x_0$ is the (pointed) fundamental group $\pi_1(X, x_0)$. In the case where $X$ is connected, and $\pi_1(X)$ is not abelian, this illustrates the fact that for a connected groupoid — whose definition you can guess — the automorphism groups of the objects are all isomorphic, but not canonically isomorphic.)

\subsubsection*{1.1.4. Example}
The abelian groups, along with group homomorphisms, form a category $\mathbf{Ab}$.

\subsubsection*{1.1.5. Important Example: Modules over a ring}
If $A$ is a ring, then the $A$-modules form a category $\mathbf{Mod}_A$. (This category has additional structure; it will be the prototypical example of an abelian category, see §1.5.) Taking $A = k$, we obtain Example 1.1.3; taking $A = \mathbb{Z}$, we obtain Example 1.1.4.

\subsubsection*{1.1.6. Example: rings}
There is a category $\mathbf{Rings}$, where the objects are rings, and the morphisms are maps of rings in the usual sense (maps of sets which respect addition and multiplication, and which send $1$ to $1$ by our conventions, §0.3).

\subsubsection*{1.1.7. Example: topological spaces}
The topological spaces, along with continuous maps, form a category $\mathbf{Top}$. The isomorphisms are homeomorphisms.

In all of the above examples, the objects of the categories were in obvious ways sets with additional structure (a concrete category, although we won’t use this terminology). This needn’t be the case, as the next example shows.

\subsubsection*{1.1.8. Example: partially ordered sets}
A partially ordered set, (or poset), is a set $S$ along with a binary relation $\geq$ on $S$ satisfying:
\begin{enumerate}
\item[(i)] $x \geq x$ (reflexivity),
\item[(ii)] $x \geq y$ and $y \geq z$ imply $x \geq z$ (transitivity), and
\item[(iii)] if $x \geq y$ and $y \geq x$ then $x = y$ (antisymmetry).
\end{enumerate}

A partially ordered set $(S, \geq)$ can be interpreted as a category whose objects are the elements of $S$, and with a single morphism from $x$ to $y$ if and only if $x \geq y$ (and no morphism otherwise).

A trivial example is $(S, \geq)$ where $x \geq y$ if and only if $x = y$. Another example is

\begin{center}
(1.1.8.1) \includegraphics[width=0.3\textwidth]{img-0.jpeg}
\end{center}

Here there are three objects. The identity morphisms are omitted for convenience, and the two non-identity morphisms are depicted. A third example is

\begin{center}
(1.1.8.2) \includegraphics[width=0.3\textwidth]{img-1.jpeg}
\end{center}

Here the “obvious” morphisms are again omitted: the identity morphisms, and the morphism from the upper left to the lower right. Similarly,

\begin{center}
\includegraphics[width=0.3\textwidth]{img-2.jpeg}
\end{center}

depicts a partially ordered set, where again, only the “generating morphisms” are depicted.

\subsubsection*{1.1.9. Example}
The category of subsets of a set, and the category of open subsets of a topological space. If $X$ is a set, then the subsets form a partially ordered set, where arrows are given by inclusion. (Be careful: you may be expecting the arrows to go the other way, because of Example 1.1.8.) Informally, if $U \subset V$, then we have exactly one morphism $U \to V$ in the category (and otherwise none). Similarly, if
X is a topological space, then the \textit{open} sets form a partially ordered set, where the maps are given by inclusions.

\subsubsection*{1.1.10. Definition}

A subcategory $\mathcal{A}$ of a category $\mathcal{B}$ has as its objects some of the objects of $\mathcal{B}$, and some of the morphisms of $\mathcal{B}$, such that the objects of $\mathcal{A}$ include the sources and targets of the morphisms of $\mathcal{A}$, and the morphisms of $\mathcal{A}$ include the identity morphisms of the objects of $\mathcal{A}$, and are preserved by composition. (For example, (1.1.8.1) is in an obvious way a subcategory of (1.1.8.2). Also, we have an obvious "inclusion" $\mathfrak{i}\colon\mathcal{A}\to\mathcal{B}$, which will soon be an example of a functor.)

\section*{1.1.11. Functors}

A covariant functor $\mathsf{F}$ from a category $\mathcal{A}$ to a category $\mathcal{B}$, denoted $\mathsf{F}\colon\mathcal{A}\to\mathcal{B}$, is the following data. It is a map of objects $\mathsf{F}\colon\operatorname{obj}(\mathcal{A})\to\operatorname{obj}(\mathcal{B})$, and for each $A_{1}$, $A_{2}\in\mathcal{A}$, and morphism $m\colon A_{1}\to A_{2}$, a morphism $\mathsf{F}(m)\colon\mathsf{F}(A_{1})\to\mathsf{F}(A_{2})$ in $\mathcal{B}$. We require that $\mathsf{F}$ preserves identity morphisms (for $A\in\mathcal{A}$, $\mathsf{F}(\operatorname{id}_{A})=\operatorname{id}_{\mathsf{F}(A)}$), and that $\mathsf{F}$ preserves composition ($\mathsf{F}(m_{2}\circ m_{1})=\mathsf{F}(m_{2})\circ\mathsf{F}(m_{1})$). (You may wish to verify that covariant functors send isomorphisms to isomorphisms.) A trivial example is the identity functor $\operatorname{id}\colon\mathcal{A}\to\mathcal{A}$, whose definition you can guess. Here are some less trivial examples.

\subsubsection*{1.1.12. Example: a forgetful functor}

Consider the functor from the category of vector spaces (over a field $\mathsf{k}$) $\operatorname{Vec}_{\mathsf{k}}$ to $\operatorname{Sets}$, that associates to each vector space its underlying set. The functor sends a linear transformation to its underlying map of sets. This is an example of a forgetful functor, where some additional structure is forgotten. Another example of a forgetful functor is $\operatorname{Mod}_{A}\to Ab$ from $A$-modules to abelian groups, remembering only the abelian group structure of the $A$-module.

\subsubsection*{1.1.13. Topological examples}

Examples of covariant functors include the fundamental group functor $\pi_{1}$, which sends a topological space $X$ with choice of a point $x_{0}\in X$ to a group $\pi_{1}(X,x_{0})$ (what are the objects and morphisms of the source category?), and the $\operatorname{ith}$ homology functor $\operatorname{Top}\to Ab$, which sends a topological space $X$ to its $\operatorname{ith}$ homology group $H_{i}(X,\mathbb{Z})$. The covariance corresponds to the fact that a (continuous) morphism of pointed topological spaces $\phi\colon X\to Y$ with $\phi(x_{0})=y_{0}$ induces a map of fundamental groups $\pi_{1}(X,x_{0})\to\pi_{1}(Y,y_{0})$, and similarly for homology groups.

\subsubsection*{1.1.14. Example}

Suppose $A$ is an object in a category $\mathcal{C}$. Then there is a functor $h^{A}\colon\mathcal{C}\to\operatorname{Sets}$ sending $B\in\mathcal{C}$ to $\operatorname{Mor}(A,B)$, and sending $f\colon B_{1}\to B_{2}$ to $\operatorname{Mor}(A,B_{1})\to\operatorname{Mor}(A,B_{2})$ described by
\[
[g\colon A\to B_{1}]\longmapsto[f\circ g\colon A\to B_{1}\to B_{2}].
\]
This seemingly silly functor ends up surprisingly being an important concept.

\subsubsection*{1.1.15. Definitions.}

If $\mathsf{F}\colon\mathcal{A}\to\mathcal{B}$ and $G\colon\mathcal{B}\to\mathcal{C}$ are covariant functors, then we define a functor $G\circ\mathsf{F}\colon\mathcal{A}\to\mathcal{C}$ (the composition of $G$ and $\mathsf{F}$) in the obvious way. Composition of functors is associative in an evident sense.

A covariant functor $\mathsf{F}\colon\mathcal{A}\to\mathcal{B}$ is faithful if for all $A,A^{\prime}\in\mathcal{A}$, the map $\operatorname{Mor}_{\mathcal{A}}(A,A^{\prime})\to\operatorname{Mor}_{\mathcal{B}}(\mathsf{F}(A),\mathsf{F}(A^{\prime}))$ is injective, and full if it is surjective. A functor that is full and faithful is fully faithful. (For various philosophical reasons, the notion of “full” functor on its own is unimportant; “fully faithful” is the useful

notion.) A subcategory $i\colon\mathcal{A}\to\mathcal{B}$ is a full subcategory if $i$ is full. (Inclusions are always faithful, so there is no need for the phrase “faithful subcategory”.) Thus a subcategory $\mathcal{A}^{\prime}$ of $\mathcal{A}$ is full if and only if for all $A,B\in\text{obj}(\mathcal{A}^{\prime})$, $\text{Mor}_{\mathcal{A}^{\prime}}(A,B)=\text{Mor}_{\mathcal{A}}(A,B)$. For example, the forgetful functor $Vec_{k}\to Sets$ is faithful, but not full; and if $A$ is a ring, the category of finitely generated $A$-modules is a full subcategory of the category $Mod_{A}$ of $A$-modules.

\subsubsection*{1.1.16. Definition}

A contravariant functor is defined in the same way as a covariant functor, except the arrows switch directions: in the above language, $F(A_{1}\to A_{2})$ is now an arrow from $F(A_{2})$ to $F(A_{1})$. (Thus $F(m_{2}\circ m_{1})=F(m_{1})\circ F(m_{2})$, not $F(m_{2})\circ F(m_{1})$.)

It is wise to state whether a functor is covariant or contravariant, unless the context makes it very clear. If it is not stated (and the context does not make it clear), the functor is often assumed to be covariant.

Sometimes people describe a contravariant functor $\mathcal{C}\to\mathcal{D}$ as a covariant functor $\mathcal{C}^{\text{opp}}\to\mathcal{D}$, where $\mathcal{C}^{\text{opp}}$ is the same category as $\mathcal{C}$ except that the arrows go in the opposite direction. Here $\mathcal{C}^{\text{opp}}$ is said to be the opposite category to $\mathcal{C}$.

One can define fullness, etc. for contravariant functors, and you should do so.

\subsubsection*{1.1.17. Linear algebra example}

If $Vec_{k}$ is the category of $k$-vector spaces (introduced in Example 1.1.3), then taking duals gives a contravariant functor $(\cdot)^{\vee}\colon Vec_{k}\to Vec_{k}$. Indeed, to each linear transformation $f\colon V\to W$, we have a dual transformation $f^{\vee}\colon W^{\vee}\to V^{\vee}$, and $(f\circ g)^{\vee}=g^{\vee}\circ f^{\vee}$.

\subsubsection*{1.1.18. Topological example}

(cf. Example 1.1.13) for those who have seen cohomology. The $i$th cohomology functor $H^{i}(\cdot,\mathbb{Z})\colon Top\to Ab$ is a contravariant functor.

\subsubsection*{1.1.19. Example}

There is a contravariant functor $Top\to Rings$ taking a topological space $X$ to the ring of real-valued continuous functions on $X$. A morphism of topological spaces $X\to Y$ (a continuous map) induces the pullback map from functions on $Y$ to functions on $X$.

\subsubsection*{1.1.20. Example (the functor of points)}

(cf. Example 1.1.14). Suppose $A$ is an object of a category $\mathcal{C}$. Then there is a contravariant functor $h_{A}\colon\mathcal{C}\to Sets$ sending $B\in\mathcal{C}$ to $\text{Mor}(B,A)$, and sending the morphism $f\colon B_{1}\to B_{2}$ to the morphism $\text{Mor}(B_{2},A)\to\text{Mor}(B_{1},A)$ via
\[
[g\colon B_{2}\to A]\longmapsto[g\circ f\colon B_{1}\to B_{2}\to A].
\]
This example initially looks weird and different, but Examples 1.1.17 and 1.1.19 may be interpreted as special cases; do you see how? What is $A$ in each case? This functor might reasonably be called the functor of maps (to $A$), but is actually known as the functor of points. We will meet this functor again in §1.2.11 and (in the category of schemes) in Definition 7.3.10.

\subsubsection*{1.1.21.}

$\star$ \textbf{Natural transformations (and natural isomorphisms) of covariant functors, and equivalences of categories.}

(This notion won’t come up in an essential way until at least Chapter 7, so you shouldn’t read this section until then.) Suppose $F$ and $G$ are two covariant functors from $\mathcal{A}$ to $\mathcal{B}$. A natural transformation of covariant functors $F\to G$ is the data of a morphism $m_{A}\colon F(A)\to G(A)$ for each $A\in\mathcal{A}$ such that for each $f\colon A\to A^{\prime}$ in $\mathcal{A}$

\begin{center}
\includegraphics[width=0.4\textwidth]{img-3.jpeg}
\end{center}

commutes. A natural isomorphism of functors is a natural transformation such that each $m_A$ is an isomorphism. (We make analogous definitions when $F$ and $G$ are both contravariant.)

The data of functors $F \colon \mathcal{A} \to \mathcal{B}$ and $F' \colon \mathcal{B} \to \mathcal{A}$ such that $F \circ F'$ is naturally isomorphic to the identity functor $\mathrm{id}_{\mathcal{B}}$ on $\mathcal{B}$ and $F' \circ F$ is naturally isomorphic to $\mathrm{id}_{\mathcal{A}}$ is said to be an equivalence of categories. The right notion of when two categories are "essentially the same" is not isomorphism (a functor giving bijections of objects and morphisms) but equivalence. Exercises 1.1.C and 1.1.D might give you some vague sense of this. Later exercises (for example, that "rings" and "affine schemes" are essentially the same, once arrows are reversed, Exercise 7.3.E) may help too.

Two examples might make this strange concept more comprehensible. The double dual of a finite-dimensional vector space $V$ is not $V$, but we learn early to say that it is canonically isomorphic to $V$. We can make that precise as follows. Let $f.d.Vec_{k}$ be the category of finite-dimensional vector spaces over $k$. Note that this category contains oodles of vector spaces of each dimension.

\subsubsection*{1.1.C. EXERCISE.} Let $(\cdot)^{\vee \vee} \colon f.d.Vec_{k} \to f.d.Vec_{k}$ be the double dual functor from the category of finite-dimensional vector spaces over $k$ to itself. Show that $(\cdot)^{\vee \vee}$ is naturally isomorphic to the identity functor on $f.d.Vec_{k}$. (Without the finite-dimensionality hypothesis, we only get a natural transformation of functors from $\operatorname{id}$ to $(\cdot)^{\vee \vee}$.)

Let $\mathcal{V}$ be the category whose objects are the $k$-vector spaces $k^n$ for each $n \geq 0$ (there is one vector space for each $n$), and whose morphisms are linear transformations. The objects of $\mathcal{V}$ can be thought of as vector spaces with bases, and the morphisms as matrices. There is an obvious functor $\mathcal{V} \to f.d.Vec_k$, as each $k^n$ is a finite-dimensional vector space.

\subsubsection*{1.1.D. EXERCISE.} Show that $\mathcal{V} \to f.d.Vec_k$ gives an equivalence of categories, by describing an "inverse" functor. (Recall that we are being cavalier about set-theoretic assumptions, see Caution 0.3.1, so feel free to simultaneously choose bases for each vector space in $f.d.Vec_k$. To make this precise, you will need to use Gödel-Bernays set theory or else replace $f.d.Vec_k$ with a very similar small category, but we won't worry about this.)

\subsubsection*{1.1.22. $\star \star$ Aside for experts}

Your argument for Exercise 1.1.D will show that (modulo set-theoretic issues) this definition of equivalence of categories is the same as another one commonly given: a covariant functor $F\colon \mathcal{A}\to \mathcal{B}$ is an equivalence of categories if it is fully faithful and every object of $\mathcal{B}$ is isomorphic to an object of the form $F(A)$ for some $A\in \mathcal{A}$ ($F$ is essentially surjective, a term we will not need).

\end{document}